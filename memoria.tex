%%%%%%%%%%%%%%%%%%%%%%%%%%%%%%%%%%%%%%%%%%%%%%%%%%%%%%%%%%%%%%%%%%%%%%%%%%%%%%%%
%% Plantilla de memoria en LaTeX para la EIF - Universidad Rey Juan Carlos
%%
%% Por Gregorio Robles <grex arroba gsyc.urjc.es>
%%     Grupo de Sistemas y Comunicaciones
%%     Escuela de Ingeniería de Fuenlabrada
%%     Universidad Rey Juan Carlos
%% (muchas ideas tomadas de Internet, colegas del GSyC, antiguos alumnos...
%%  etc. Muchas gracias a todos)
%%
%% La última versión de esta plantilla está siempre disponible en:
%%     https://github.com/gregoriorobles/plantilla-memoria
%%
%% Para obtener PDF, ejecuta en la shell:
%%   make
%% (las imágenes deben ir en PNG o JPG)

%%%%%%%%%%%%%%%%%%%%%%%%%%%%%%%%%%%%%%%%%%%%%%%%%%%%%%%%%%%%%%%%%%%%%%%%%%%%%%%%

\documentclass[a4paper, 12pt]{book}
%\usepackage[T1]{fontenc}

\usepackage[a4paper, left=2.5cm, right=2.5cm, top=3cm, bottom=3cm]{geometry}
\usepackage{times}
\usepackage[utf8]{inputenc}
\usepackage[spanish]{babel} % Comenta esta línea si tu memoria es en inglés
\usepackage{url}
%\usepackage[dvipdfm]{graphicx}
\usepackage{graphicx}
\usepackage{float}  %% H para posicionar figuras
\usepackage[nottoc, notlot, notlof, notindex]{tocbibind} %% Opciones de índice
\usepackage{latexsym}  %% Logo LaTeX

\title{Memoria del Proyecto}
\author{Nombre del autor}

\renewcommand{\baselinestretch}{1.5}  %% Interlineado

\begin{document}
	
	\renewcommand{\refname}{Bibliografía}  %% Renombrando
	\renewcommand{\appendixname}{Apéndice}
	
	
	%%%%%%%%%%%%%%%%%%%%%%%%%%%%%%%%%%%%%%%%%%%%%%%%%%%%%%%%%%%%%%%%%%%%%%%%%%%%%%%%
	% PORTADA
	
	\begin{titlepage}
		\begin{center}
			\includegraphics[scale=0.6]{img/URJ_logo_Color_POS.png}
			
			\vspace{1.75cm}
			
			\LARGE
			ESCUELA DE INGENIERÍA DE FUENLABRADA
			\vspace{1cm}
			
			\LARGE
			GRADO EN INGENIERÍA EN TECNOLOGÍAS DE LA TELECOMUNICACIÓN
			
			\vspace{1cm}
			\LARGE
			\textbf{TRABAJO FIN DE GRADO}
			
			\vspace{1cm}
			
			\Large
			CONFIGURACIÓN Y PRUEBAS DE UN PLANO DE CONTROL IN-BAND SDN/OPENFLOW: MAQUETA CON SWITCHES ETHERNET  
			
			\vspace{2cm}
			
			\large
			Autor : Luis Haro Peña \\
			Tutor : Pedro de las Heras Quirós\\
			Cotutor: Eva María Castro Barbero
			\vspace{1cm}
			
			\large
			Curso académico 2022/2023
			
		\end{center}
	\end{titlepage}
	
	\newpage
	\mbox{}
	\thispagestyle{empty} % para que no se numere esta pagina
	
	
	
	%%%%%%%%%%%%%%%%%%%%%%%%%%%%%%%%%%%%%%%%%%%%%%%%%%%%%%%%%%%%%%%%%%%%%%%%%%%%%%%%
	%%%% Para firmar
	\clearpage
	\pagenumbering{gobble}
	\chapter*{}
	
	\vspace{-4cm}
	\begin{center}
		\LARGE
		\textbf{Trabajo Fin de Grado}
		
		\vspace{1cm}
		\large
		Configuración y Pruebas de un Plano de Control In-Band SDN/Openflow: Maqueta con Switches Ethernet
		
		\vspace{1cm}
		\large
		\textbf{Autor :} Luis Haro Peña \\
		\textbf{Tutor :} Pedro de las Heras Quirós
		
	\end{center}
	
	\vspace{1cm}
	La defensa del presente Proyecto Fin de Carrera se realizó el día \qquad$\;\,$ de \qquad\qquad\qquad\qquad \newline de 2023, siendo calificada por el siguiente tribunal:
	
	
	\vspace{0.5cm}
	\textbf{Presidente:}
	
	\vspace{1.2cm}
	\textbf{Secretario:}
	
	\vspace{1.2cm}
	\textbf{Vocal:}
	
	
	\vspace{1.2cm}
	y habiendo obtenido la siguiente calificación:
	
	\vspace{1cm}
	\textbf{Calificación:}
	
	
	\vspace{1cm}
	\begin{flushright}
		Fuenlabrada, a \qquad$\;\,$ de \qquad\qquad\qquad\qquad de 2023
	\end{flushright}
	
	
	%%%%%%%%%%%%%%%%%%%%%%%%%%%%%%%%%%%%%%%%%%%%%%%%%%%%%%%%%%%%%%%%%%%%%%%%%%%%%%%%
	%%%% Agradecimientos
	
	\chapter*{Agradecimientos}
	%\addcontentsline{toc}{chapter}{Agradecimientos} % si queremos que aparezca en el índice
	\markboth{AGRADECIMIENTOS}{AGRADECIMIENTOS} % encabezado 
	
	Estimada familia, pareja, amigos y seres queridos,
	
	Hoy quiero tomar un momento para expresar mi más profundo agradecimiento por todo su apoyo y amor incondicional durante mi trayectoria académica y, en particular, en la culminación de mi TFG. Vuestra presencia constante, aliento y comprensión han sido fundamentales en mi camino hacia el éxito.
	
	En primer lugar, quiero agradecer a mis padres, quienes me han brindado un apoyo inquebrantable desde el principio. Vuestra dedicación, paciencia y sacrificios han sido el pilar sobre el cual he construido mi educación. Gracias por estar siempre ahí, incluso en los momentos más difíciles.
		
	A mi pareja, quiero expresar mi más profundo agradecimiento por tu amor, paciencia y apoyo a lo largo de este viaje académico. Tú has sido mi roca, mi fuente de inspiración y mi motivación constante. Tus palabras de aliento y tu presencia reconfortante me han dado fuerzas para superar cualquier obstáculo que se haya presentado en el camino. Gracias por ser mi compañera de vida y por estar a mi lado en cada paso del camino.
	
	Con gratitud,
	Luis.
	
	%%%%%%%%%%%%%%%%%%%%%%%%%%%%%%%%%%%%%%%%%%%%%%%%%%%%%%%%%%%%%%%%%%%%%%%%%%%%%%%%
	%%%% Resumen
	
	\chapter*{Resumen}
	%\addcontentsline{toc}{chapter}{Resumen} % si queremos que aparezca en el índice
	\markboth{RESUMEN}{RESUMEN} % encabezado
	
	Aquí viene un resumen del proyecto.
	Ha de constar de tres o cuatro párrafos, donde se presente de manera clara y concisa de qué va el proyecto. 
	Han de quedar respondidas las siguientes preguntas:
	
	\begin{itemize}
		\item ¿De qué va este proyecto? ¿Cuál es su objetivo principal?
		\item ¿Cómo se ha realizado? ¿Qué tecnologías están involucradas?
		\item ¿En qué contexto se ha realizado el proyecto? ¿Es un proyecto dentro de un marco general?
	\end{itemize}
	
	Lo mejor es escribir el resumen al final.
	
	%%%%%%%%%%%%%%%%%%%%%%%%%%%%%%%%%%%%%%%%%%%%%%%%%%%%%%%%%%%%%%%%%%%%%%%%%%%%%%%%
	%%%% Resumen en inglés
	
	\chapter*{Summary}
	%\addcontentsline{toc}{chapter}{Summary} % si queremos que aparezca en el índice
	\markboth{SUMMARY}{SUMMARY} % encabezado
	
	Here comes a translation of the ``Resumen'' into English. 
	Please, double check it for correct grammar and spelling.
	As it is the translation of the ``Resumen'', which is supposed to be written at the end, this as well should be filled out just before submitting.
	
	
	%%%%%%%%%%%%%%%%%%%%%%%%%%%%%%%%%%%%%%%%%%%%%%%%%%%%%%%%%%%%%%%%%%%%%%%%%%%%%%%%
	%%%%%%%%%%%%%%%%%%%%%%%%%%%%%%%%%%%%%%%%%%%%%%%%%%%%%%%%%%%%%%%%%%%%%%%%%%%%%%%%
	% ÍNDICES %
	%%%%%%%%%%%%%%%%%%%%%%%%%%%%%%%%%%%%%%%%%%%%%%%%%%%%%%%%%%%%%%%%%%%%%%%%%%%%%%%%
	
	% Las buenas noticias es que los índices se generan automáticamente.
	% Lo único que tienes que hacer es elegir cuáles quieren que se generen,
	% y comentar/descomentar esa instrucción de LaTeX.
	
	%%%% Índice de contenidos
	\tableofcontents 
	%%%% Índice de figuras
	\cleardoublepage
	%\addcontentsline{toc}{chapter}{Lista de figuras} % para que aparezca en el indice de contenidos
	\listoffigures % indice de figuras
	%%%% Índice de tablas
	%\cleardoublepage
	%\addcontentsline{toc}{chapter}{Lista de tablas} % para que aparezca en el indice de contenidos
	%\listoftables % indice de tablas
	
	
	%%%%%%%%%%%%%%%%%%%%%%%%%%%%%%%%%%%%%%%%%%%%%%%%%%%%%%%%%%%%%%%%%%%%%%%%%%%%%%%%
	%%%%%%%%%%%%%%%%%%%%%%%%%%%%%%%%%%%%%%%%%%%%%%%%%%%%%%%%%%%%%%%%%%%%%%%%%%%%%%%%
	% INTRODUCCIÓN %
	%%%%%%%%%%%%%%%%%%%%%%%%%%%%%%%%%%%%%%%%%%%%%%%%%%%%%%%%%%%%%%%%%%%%%%%%%%%%%%%%
	
	\clearpage
	\chapter{Introducción}
	\label{sec:intro} % etiqueta para poder referenciar luego en el texto con ~\ref{sec:intro}
	\pagenumbering{arabic} % para empezar la numeración de página con números
	
	
	Escribir una breve introducción del TFG.
	
	\section{Contexto}
	\label{sec:contexto}
	
	Este trabajo consiste en el análisis y comprensión de un plano de control In-Band SDN/Openflow en varios escenarios con diferente número de controladores.
	
	 Se realiza debido a....
	

	
	%%%%%%%%%%%%%%%%%%%%%%%%%%%%%%%%%%%%%%%%%%%%%%%%%%%%%%%%%%%%%%%%%%%%%%%%%%%%%%%%
	%%%%%%%%%%%%%%%%%%%%%%%%%%%%%%%%%%%%%%%%%%%%%%%%%%%%%%%%%%%%%%%%%%%%%%%%%%%%%%%%
	% OBJETIVOS %
	%%%%%%%%%%%%%%%%%%%%%%%%%%%%%%%%%%%%%%%%%%%%%%%%%%%%%%%%%%%%%%%%%%%%%%%%%%%%%%%%
	
	\cleardoublepage % empezamos en página impar
	\chapter{Objetivos} % título del capítulo (se muestra)
	\label{chap:objetivos} % identificador del capítulo (no se muestra, es para poder referenciarlo)
	
	\section{Objetivo general} % título de sección (se muestra)
	\label{sec:objetivo-general} % identificador de sección (no se muestra, es para poder referenciarla)
	
	En el presente apartado se va a dedicar al establecimiento del objetivo principal para la realización de la investigación. 
	
	Como he mencionado anteriormente, el presente Trabajo de Fin de Grado consiste en configurar, analizar y representar varios escenarios utilizando In-band/SDN-Openflow para ver su comportamiento. Se buscará principalmente, conocer cuanto tarda cada switch en estar manejado por un controlador, comparando como de rentable es en cuanto a tiempo que se añadan mas controladores.
	
	
	\section{Objetivos específicos}
	\label{sec:objetivos-especificos}
	
	Para la realización de esta investigación, pretendemos cumplir los siguientes objetivos específicos: 
	
	\begin{enumerate}
		\item 	Realizar 10 experimentos en cada escenario para conseguir diferentes resultados.
		\item  	Analizar y comparar los resultados obtenidos.
		\item 	Estudiar los resultados con el objetivo de llegar a una conclusión.
		
	\end{enumerate}
	
 	Se va a analizar primero el escenario ~\ref{figura:bucle4} y el mismo escenario con 2 controladores ~\ref{figura:2controllers}
	
		\begin{figure}
			\centering
			\includegraphics[width=16cm, keepaspectratio]{img/bucle4}
			\caption{Primer escenario con un controlador}
			\label{figura:bucle4}
		\end{figure}
	
		\begin{figure}
			\centering
			\includegraphics[width=16cm, keepaspectratio]{img/2controllers}
			\caption{Primer escenario con 2 controladores}
			\label{figura:2controllers}
		\end{figure}
	
		Posteriormente, se analizará un escenario mas complejo ~\ref{figura:mesh} y el mismo escenario con 4 controladores en vez de 1 ~\ref{figura:mesh4c}.
	
		\begin{figure}
			\centering
			\includegraphics[width=16cm, keepaspectratio]{img/mesh}
			\caption{Segundo escenario con 1 controlador}
			\label{figura:mesh}
		\end{figure}
	
		\begin{figure}
			\centering
			\includegraphics[width=16cm, keepaspectratio]{img/mesh4c}
			\caption{Segundo escenario con 4 controladores}
			\label{figura:mesh4c}
		\end{figure}
	
	\cleardoublepage
	\section{Planificación temporal}
	\label{sec:planificacion-temporal}
	
	El TFG se ha realizado durante mas de un año, de forma intermitente, principalmente en fines de semana y festivos. Durante todo este tiempo, hemos estado intercambiando correos, y teniendo varias reuniones de seguimiento, principalmente porque un requisito importante de esteTFG era entender perfectamente lo que ocurría en el escenario, viendo así capas muchos mas bajas de las que finalmente he tenido que usar para realizar los experimentos.
	Además ha habido varios problemas con las herramientas utilizadas que han ralentizado mucho el trabajo, con periodos de inactividad de varios meses intentando solventar los problemas.
	
	
	%%%%%%%%%%%%%%%%%%%%%%%%%%%%%%%%%%%%%%%%%%%%%%%%%%%%%%%%%%%%%%%%%%%%%%%%%%%%%%%%
	%%%%%%%%%%%%%%%%%%%%%%%%%%%%%%%%%%%%%%%%%%%%%%%%%%%%%%%%%%%%%%%%%%%%%%%%%%%%%%%%
	% ESTADO DEL ARTE %
	%%%%%%%%%%%%%%%%%%%%%%%%%%%%%%%%%%%%%%%%%%%%%%%%%%%%%%%%%%%%%%%%%%%%%%%%%%%%%%%%
	
	\cleardoublepage
	\chapter{Estado del arte}
	\label{chap:estado}
	
	En este capítulo se detallan las tecnologías utilizadas en el desarrollo del proyecto.	
	
	\section{Mininet} 
	\label{sec:mininet}
	
	Mininet es una plataforma de emulación de redes de código abierto que permite crear una red virtual utilizando equipos de red virtuales (Switches y hosts) dentro de un entorno Linux. Proporciona un entorno de pruebas realista y escalable para experimentar, desarrollar y probar protocolos de red, aplicaciones y sistemas distribuidos.
	
	Algunas características importantes de Mininet incluyen:
	
	\begin{enumerate}
		\item 	Emulación de red: Mininet permite crear una topología de red virtual con switches Open vSwitch (OVS) y hosts emulados. Puedes definir la topología de red y las características de los dispositivos de red utilizando scripts en Python.
		\item 	Entorno aislado: Cada instancia de Mininet se ejecuta en su propio espacio de nombres (namespace) y utiliza recursos de red virtuales, lo que proporciona un entorno aislado y seguro para realizar experimentos.
		\item 	Integración con controladores SDN: Mininet se integra fácilmente con controladores SDN como OpenDaylight, ONOS o Ryu. Esto permite probar y desarrollar aplicaciones y protocolos de red definidos por software utilizando un entorno emulado.
		\item 	Simulación de tráfico: Mininet permite generar y simular tráfico de red en la topología emulada. Esto es útil para evaluar el rendimiento de la red, analizar el comportamiento de los protocolos y probar la tolerancia a fallos.
		\item   Extensibilidad: Mininet es altamente extensible y se puede personalizar según las necesidades del usuario. Es posible agregar nuevos tipos de nodos de red, desarrollar controladores personalizados y utilizar módulos de extensión para ampliar su funcionalidad.
	\end{enumerate}

	
	En resumen, Mininet es una herramienta muy útil para la experimentación y desarrollo de redes. Proporciona un entorno de red virtualizado y aislado, lo que permite probar y evaluar diversas configuraciones de red y escenarios de manera eficiente y segura.
	
	Mininet es la herramienta con la cual se han realizado los experimentos y simulado los escenarios.
	
	\section{OpenFlow}
	\label{sec:openflow}
	
	
	OpenFlow es un protocolo de comunicación y una arquitectura de red que permite la separación del plano de control y el plano de datos en redes de computadoras. Fue desarrollado por la Open Networking Foundation (ONF) y se ha convertido en un estándar abierto utilizado en redes definidas por software (SDN).
	
	En una red SDN con OpenFlow, el plano de control se encuentra centralizado en un controlador, mientras que el plano de datos está distribuido en los dispositivos de red. Los dispositivos de red compatibles con OpenFlow, como los conmutadores, actúan como ejecutores de las instrucciones proporcionadas por el controlador central.
	
	El protocolo OpenFlow establece una comunicación entre el controlador y los dispositivos de red mediante mensajes definidos. El controlador puede enviar instrucciones a los dispositivos de red, como reglas de enrutamiento o de flujo, y recibir información sobre el estado de la red, como estadísticas de tráfico. Esto permite una gestión centralizada y programable de la red, lo que facilita la implementación de políticas de red, la optimización del tráfico y la resolución de problemas.
	
	Algunas de las ventajas de OpenFlow y SDN incluyen la flexibilidad y la capacidad de adaptación de la red a las necesidades cambiantes, la simplificación de la administración y la configuración de la red, y la posibilidad de implementar nuevas funcionalidades y servicios de manera más rápida y eficiente.
	
	OpenFlow es el protocolo utilizado para la realización de las pruebas, y en definitiva, es su comportamiento lo que se buscaba entender.
	
	\section{Wireshark} 
	\label{sec:wireshark}
	
	Wireshark es una herramienta de análisis de redes de código abierto y gratuita. Permite capturar y examinar el tráfico de red en tiempo real, así como analizar y visualizar datos de capturas previas. Wireshark es ampliamente utilizado por administradores de redes, profesionales de seguridad y desarrolladores de protocolos para resolver problemas de red, realizar análisis de tráfico y depurar aplicaciones.
	
	Algunas características clave de Wireshark incluyen:
	
	\begin{enumerate}
		\item 	Captura de paquetes: Wireshark puede capturar y analizar paquetes en tiempo real de diversas interfaces de red, como Ethernet, Wi-Fi y Bluetooth. Permite seleccionar las interfaces de red específicas de las que se desea capturar el tráfico.	
		\item 	Análisis detallado: Wireshark proporciona una vista detallada de cada paquete capturado, mostrando información como direcciones IP de origen y destino, puertos, protocolos, datos de encabezado y carga útil. Permite filtrar y buscar paquetes por varios criterios, lo que facilita el análisis de tráfico específico.
		\item 	Decodificación de protocolos: Wireshark es capaz de decodificar y analizar una amplia gama de protocolos de red, incluyendo TCP/IP, HTTP, DNS, SSH, FTP, entre otros. Puede reconstruir sesiones y mostrar la secuencia de intercambio de paquetes entre los dispositivos de la red.
		\item 	Estadísticas y gráficos: Wireshark proporciona estadísticas y gráficos sobre el tráfico de red capturado, como la cantidad de paquetes, la distribución de protocolos y el ancho de banda utilizado. Esto puede ayudar a identificar patrones de tráfico, detectar anomalías y optimizar el rendimiento de la red.
		\item   Soporte multiplataforma: Wireshark es compatible con varios sistemas operativos, incluyendo Windows, macOS y Linux. Además, está disponible en varios idiomas, lo que facilita su uso en diferentes entornos.
	\end{enumerate}
	
	
	Wireshark es una herramienta poderosa para el análisis de redes y resolución de problemas. Su interfaz intuitiva y sus capacidades de filtrado y análisis detallado lo convierten en una opción popular tanto para usuarios principiantes como para expertos en redes.
	
	Se ha utilizado sobretodo en las fases iniciales del TFG, ya que con esta herramienta podía entender perfectamente lo que pasaba al mas bajo nivel entre los switches y los controladores, viendo todo tipo de tráfico y comunicación.	
	
	\section{Python} 
	\label{sec:python}
	
	Python es un lenguaje de programación de alto nivel, interpretado y de propósito general. 
	
	Python es ampliamente utilizado en una variedad de campos, como desarrollo de software, ciencia de datos, aprendizaje automático, automatización de tareas, desarrollo web y más. Su popularidad se debe a su facilidad de uso, flexibilidad y su enfoque en la legibilidad del código, lo que lo convierte en un lenguaje ideal tanto para principiantes como para desarrolladores experimentados.
	
	Se ha utilizado en la realización de scripts. Por lo que también ha sido clave a la hora de entender el funcionamiento de dichos scripts y poder realizar modificaciones.
	
	%%%%%%%%%%%%%%%%%%%%%%%%%%%%%%%%%%%%%%%%%%%%%%%%%%%%%%%%%%%%%%%%%%%%%%%%%%%%%%%%
	%%%%%%%%%%%%%%%%%%%%%%%%%%%%%%%%%%%%%%%%%%%%%%%%%%%%%%%%%%%%%%%%%%%%%%%%%%%%%%%%
	% DISEÑO E IMPLEMENTACIÓN %
	%%%%%%%%%%%%%%%%%%%%%%%%%%%%%%%%%%%%%%%%%%%%%%%%%%%%%%%%%%%%%%%%%%%%%%%%%%%%%%%%
	
	\cleardoublepage
	\chapter{Diseño e implementación}
	\label{sec:diseno}
	 
	 No se si aplica.
	
	
	
	%%%%%%%%%%%%%%%%%%%%%%%%%%%%%%%%%%%%%%%%%%%%%%%%%%%%%%%%%%%%%%%%%%%%%%%%%%%%%%%%
	%%%%%%%%%%%%%%%%%%%%%%%%%%%%%%%%%%%%%%%%%%%%%%%%%%%%%%%%%%%%%%%%%%%%%%%%%%%%%%%%
	% EXPERIMENTOS Y VALIDACIÓN %
	%%%%%%%%%%%%%%%%%%%%%%%%%%%%%%%%%%%%%%%%%%%%%%%%%%%%%%%%%%%%%%%%%%%%%%%%%%%%%%%%
	
	\cleardoublepage
	\chapter{Experimentos y validación}
	\label{chap:experimentos}
	
 	Se va a realizar una tanda de 10 pruebas por cada escenario, comparando los resultados en cuanto a tiempo de conexión de los switches con el o los controladores, comparando así como afecta a un escenario el añadir nuevos controladores.
	

	%%%%%%%%%%%%%%%%%%%%%%%%%%%%%%%%%%%%%%%%%%%%%%%%%%%%%%%%%%%%%%%%%%%%%%%%%%%%%%%%
	%%%%%%%%%%%%%%%%%%%%%%%%%%%%%%%%%%%%%%%%%%%%%%%%%%%%%%%%%%%%%%%%%%%%%%%%%%%%%%%%
	% RESULTADOS %
	%%%%%%%%%%%%%%%%%%%%%%%%%%%%%%%%%%%%%%%%%%%%%%%%%%%%%%%%%%%%%%%%%%%%%%%%%%%%%%%%
	
	\clearpage
	\chapter{Resultados}
	\label{chap:resultados}
	\section{Resultados escenario 1} 
	\label{sec:resultEsc1}
 
 		Para el primer caso, el visto en ~\ref{figura:bucle4} y ~\ref{figura:2controllers}. Hemos obtenido los siguientes resultados:
 	
 	\begin{figure}[H]
 		\centering
 		\includegraphics[width=16cm, keepaspectratio]{img/comparativabucle4}
 		\caption{Comparativa de tiempos en el primer escenario}
 		\label{figura:comparativabucle4}
 	\end{figure}
 	
 	\begin{figure}[H]
 		\centering
 		\includegraphics[width=16cm, keepaspectratio]{img/comparativamediasbucle}
 		\caption{Comparativa de media de tiempos en el primer escenario}
 		\label{figura:mediabucle4}
 	\end{figure}
 	
 	Con estos resultados, podemos confirmar que la agregación de nuevos controladores a un escenario reducirá los tiempos máximos de conexión. La media de mejora en este escenario es de 0.0546 segundos, siendo esto una mejora del 14.59 \% al añadir un controlador extra al escenario.
 	
 	 	
 	Para el caso de 1 controlador, los esquemas se han ido conectando de las siguientes formas:
 	
 	\begin{figure}[H]
 		\centering
 		\includegraphics[width=16cm, keepaspectratio]{img/escenario1_1c_1}
 		\caption{Orden de conexión de los experimentos 1, 3, 4, 5, 6 ,7, 8, 10}
 		\label{figura:escenario1_1c_1}
 	\end{figure}
 	
 	\begin{figure}[H]
 		\centering
 		\includegraphics[width=16cm, keepaspectratio]{img/escenario1_1c_2}
 		\caption{Orden de conexión de los experimentos 2, 9}
 		\label{figura:escenario1_1c_2}
 	\end{figure}
 	
 	Para el caso de 2 controladores, como era de esperar es bastante diferente, quedando así:
 	
 	\begin{figure}[H]
 		\centering
 		\includegraphics[width=16cm, keepaspectratio]{img/escenario1_2c_1}
 		\caption{Orden de conexión de los experimentos 1, 4}
 		\label{figura:escenario1_2c_1}
 	\end{figure}
 	
 	\begin{figure}[H]
 		\centering
 		\includegraphics[width=16cm, keepaspectratio]{img/escenario1_2c_2}
 		\caption{Orden de conexión de los experimentos 2, 9}
 		\label{figura:escenario1_2c_2}
 	\end{figure}
 	
 	\begin{figure}[H]
 		\centering
 		\includegraphics[width=16cm, keepaspectratio]{img/escenario1_2c_3}
 		\caption{Orden de conexión del experimento 3}
 		\label{figura:escenario_2c_3}
 	\end{figure}
 	
 	\begin{figure}[H]
 		\centering
 		\includegraphics[width=16cm, keepaspectratio]{img/escenario1_2c_4}
 		\caption{Orden de conexión del experimento 5}
 		\label{figura:escenario1_2c_4}
 	\end{figure}
 	
 	\begin{figure}[H]
 		\centering
 		\includegraphics[width=16cm, keepaspectratio]{img/escenario1_2c_5}
 		\caption{Orden de conexión del experimento 6}
 		\label{figura:escenario1_2c_5}
 	\end{figure}
 	
 	\begin{figure}
 		\centering
 		\includegraphics[width=16cm, keepaspectratio]{img/escenario1_2c_6}
 		\caption{Orden de conexión del experimento 7}
 		\label{figura:escenario1_2c_6}
 	\end{figure}
 	
 	\begin{figure}[H]
 		\centering
 		\includegraphics[width=16cm, keepaspectratio]{img/escenario1_2c_7}
 		\caption{Orden de conexión del experimento 8}
 		\label{figura:escenario1_2c_7}
 	\end{figure}
 	
 	\begin{figure}[H]
 		\centering
 		\includegraphics[width=16cm, keepaspectratio]{img/escenario1_2c_8}
 		\caption{Orden de conexión del experimento 10}
 		\label{figura:escenario1_2c_8}
 	\end{figure}
 	
 	\section{Resultados escenario 2} 
 	\label{sec:resultEsc2}
 	
 	Para el segundo caso, se trabaja con un escenario mucho mas grande, el visto en ~\ref{figura:mesh} y ~\ref{figura:mesh4c}. Hemos obtenido los siguientes resultados:
 	
 	\begin{figure}[H]
 		\centering
 		\includegraphics[width=16cm, keepaspectratio]{img/comparativamesh}
 		\caption{Comparativa de tiempos en el segundo escenario}
 		\label{figura:comparativamesh}
 	\end{figure}
 	
 	\begin{figure}[H]
 		\centering
 		\includegraphics[width=16cm, keepaspectratio]{img/comparativamediamesh}
 		\caption{Comparativa de media de tiempos en el segundo escenario}
 		\label{figura:mediamesh}
 	\end{figure}
 	
	Con estos resultados, podemos confirmar lo visto en el escenario 1, ya que además en este escenario, al ser mas grande y complejo, la disminución es mas notoria, ya que el tiempo máximo de conexión en los switches se ha reducido en 0.1464 segundos, siendo el sistema un 26.28 \% mas rápido con 4 controladores que con 1.
	
	%%%%%%%%%%%%%%%%%%%%%%%%%%%%%%%%%%%%%%%%%%%%%%%%%%%%%%%%%%%%%%%%%%%%%%%%%%%%%%%%
	%%%%%%%%%%%%%%%%%%%%%%%%%%%%%%%%%%%%%%%%%%%%%%%%%%%%%%%%%%%%%%%%%%%%%%%%%%%%%%%%
	% CONCLUSIONES %
	%%%%%%%%%%%%%%%%%%%%%%%%%%%%%%%%%%%%%%%%%%%%%%%%%%%%%%%%%%%%%%%%%%%%%%%%%%%%%%%%
	
	\clearpage
	\chapter{Conclusiones}
	\label{chap:conclusiones}
	
	Como conclusión de lo visto en los diferentes experimentos, podemos confirmar que añadir un controlador a un escenario, va a ayudar a la hora de minimizar tiempos de conexión. Además, dicho tiempo mejorará cuanto mas lejos estén los nuevos controladores de los actuales del sistema, ya que la conexión tendrá que pasar por menos switches, y por tanto, se realizarán menos llamadas.
	
	\section{Consecución de objetivos}
	\label{sec:consecucion-objetivos}
	
	Esta sección es la sección espejo de las dos primeras del capítulo de objetivos, donde se planteaba el objetivo general y se elaboraban los específicos.
	
	\section{Aplicación de lo aprendido}
	\label{sec:aplicacion}
	
	\begin{enumerate}
		\item Arquitectura de Redes de Ordenadores
		\item Sistemas Telemáticos
	\end{enumerate}
	
	
	\section{Lecciones aprendidas}
	\label{sec:lecciones_aprendidas}
	
	A raíz de empezar a investigar para poder realizar el presente TFG hemos podido adquirir conocimientos en lo que respecta a:
	
	\begin{enumerate}
		\item OpenFlow. Al ser el elemento principal de esta investigación, la mayor parte de conocimiento que he adquirido ha sido sobre el protocolo de comunicación OpenFlow. El protocolo OpenFlow establece una comunicación entre el controlador y los dispositivos de red mediante mensajes definidos. El controlador puede enviar instrucciones a los dispositivos de red, como reglas de enrutamiento o de flujo, y recibir información sobre el estado de la red, como estadísticas de tráfico. Esto permite una gestión centralizada y programable de la red, lo que facilita la implementación de políticas de red, la optimización del tráfico y la resolución de problemas.
		\item LaTeX, ya que la memoria se ha realizado íntegramente con LaTeX. Con la ayuda de la plantilla hecha por el profesor Gregorio Robles y sus indicaciones, ha sido bastante sencillo realizar la documentación con dicha herramienta.
		\item Python. Pese a que el trabajo en si no tenía programación, he tenido que aprender las bases del lenguaje para entender los Scripts que se han utilizado tanto para definir los escenarios como para la realización de las pruebas.
		\item Desarrollar actitudes de investigación.
		
	\end{enumerate}
	
	
	\section{Trabajos futuros}
	\label{sec:trabajos_futuros}
	
	Tras realizar el presente TFG y visto lo aprendido y analizado el contenido del mismo, considero que sería una buena opción a tener en cuenta el optimizar los escenarios del sistema, creando lógica en los controladores, con el objetivo de evitar saturar determinados switches. Asimismo opino que podría ser interesante investigar sobre hasta que punto es factible añadir controladores extra a un escenario, ya que posiblemente llegue un punto que la mejora no sea lo suficientemente satisfactoria.
	
	
	%%%%%%%%%%%%%%%%%%%%%%%%%%%%%%%%%%%%%%%%%%%%%%%%%%%%%%%%%%%%%%%%%%%%%%%%%%%%%%%%
	%%%%%%%%%%%%%%%%%%%%%%%%%%%%%%%%%%%%%%%%%%%%%%%%%%%%%%%%%%%%%%%%%%%%%%%%%%%%%%%%
	% APÉNDICE(S) %
	%%%%%%%%%%%%%%%%%%%%%%%%%%%%%%%%%%%%%%%%%%%%%%%%%%%%%%%%%%%%%%%%%%%%%%%%%%%%%%%%
	
	\cleardoublepage
	
	%%%%%%%%%%%%%%%%%%%%%%%%%%%%%%%%%%%%%%%%%%%%%%%%%%%%%%%%%%%%%%%%%%%%%%%%%%%%%%%%
	%%%%%%%%%%%%%%%%%%%%%%%%%%%%%%%%%%%%%%%%%%%%%%%%%%%%%%%%%%%%%%%%%%%%%%%%%%%%%%%%
	% BIBLIOGRAFIA %
	%%%%%%%%%%%%%%%%%%%%%%%%%%%%%%%%%%%%%%%%%%%%%%%%%%%%%%%%%%%%%%%%%%%%%%%%%%%%%%%%
	
	\cleardoublepage
	
	% Las siguientes dos instrucciones es todo lo que necesitas
	% para incluir las citas en la memoria
	\bibliographystyle{abbrv}
	\bibliography{memoria}  % memoria.bib es el nombre del fichero que contiene
	% las referencias bibliográficas. Abre ese fichero y mira el formato que tiene,
	% que se conoce como BibTeX. Hay muchos sitios que exportan referencias en
	% formato BibTeX. Prueba a buscar en http://scholar.google.com por referencias
	% y verás que lo puedes hacer de manera sencilla.
	% Más información: 
	% http://texblog.org/2014/04/22/using-google-scholar-to-download-bibtex-citations/
	
\end{document}
